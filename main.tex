%% The MIT License (MIT)
%%
%% Copyright (c) 2015 Daniil Belyakov
%%
%% Permission is hereby granted, free of charge, to any person obtaining a copy
%% of this software and associated documentation files (the "Software"), to deal
%% in the Software without restriction, including without limitation the rights
%% to use, copy, modify, merge, publish, distribute, sublicense, and/or sell
%% copies of the Software, and to permit persons to whom the Software is
%% furnished to do so, subject to the following conditions:
%%
%% The above copyright notice and this permission notice shall be included in all
%% copies or substantial portions of the Software.
%%
%% THE SOFTWARE IS PROVIDED "AS IS", WITHOUT WARRANTY OF ANY KIND, EXPRESS OR
%% IMPLIED, INCLUDING BUT NOT LIMITED TO THE WARRANTIES OF MERCHANTABILITY,
%% FITNESS FOR A PARTICULAR PURPOSE AND NONINFRINGEMENT. IN NO EVENT SHALL THE
%% AUTHORS OR COPYRIGHT HOLDERS BE LIABLE FOR ANY CLAIM, DAMAGES OR OTHER
%% LIABILITY, WHETHER IN AN ACTION OF CONTRACT, TORT OR OTHERWISE, ARISING FROM,
%% OUT OF OR IN CONNECTION WITH THE SOFTWARE OR THE USE OR OTHER DEALINGS IN THE
%% SOFTWARE.

% The font could be set to Windows-specific Calibri by using the 'calibri' option
\documentclass[]{mcdowellcv}
\usepackage{hyperref}

% For mathematical symbols
\usepackage{amsmath}

% Set applicant's personal data for header
\name{Shanuj Shekhar}
\renewcommand{\baselinestretch}{1.1}

\begin{document}

	% Print the header
	\makeheader
    \centering Phone : +1 (650) 309-3713 \hspace{1mm} \bullet \hspace{2mm} Email : shanuj.shekhar@stonybrook.edu \hspace{1mm} \bullet
    \hspace{1mm} LinkedIn: \href{https://www.linkedin.com/in/shanujshekhar/}{\underline{linkedin.com/in/shanujshekhar/}} \newline \bullet \hspace{1mm} GitHub: \hspace{1mm}\href{https://github.com/shanujshekhar/}{\underline{github.com/shanujshekhar/}} \bullet \hspace{1mm} Website: \href{https://shanujshekhar.github.io/}{\underline{shanujshekhar.github.io}}

	% Print the content
	\begin{cvsection}{Education}
		\begin{cvsubsection}{SUNY Stony Brook}{}{Aug 2019 -- Dec 2020 \textit{(Expected)}}
			\begin{itemize}
				\item Masters in Computer and Information Science | State University of New York at Stony Brook | \textbf{GPA: 3.61/4.00}
			\end{itemize}
		\end{cvsubsection}
        \begin{cvsubsection}{NIT Jalandhar}{}{Aug 2015 -- May 2019}
			\begin{itemize}
				\item Bachelors of Technology in Computer Science and Engineering | Dr. B.R.Ambedkar National Institute of Technology | \textbf{GPA: 8.53/10.00}
			\end{itemize}
		\end{cvsubsection}
	\end{cvsection}
	
	\begin{cvsection}{Languages and Technologies}
		\begin{cvsubsection}{}{}{}	
			\begin{itemize}
				\item Code mainly in \textbf{Java \& Python}; Proficient in \textbf{JavaScript, HTML, CSS, D3.js}; Familiar with \textbf{C/C++, C\#}; 
				\item \textbf{Machine Learning \& Data Analysis:} PyTorch, Tensorflow, Numpy/Scipy, Pandas, Scikit-learn, Matplotlib, OpenCV, nltk
				\item \textbf{Other Tools:} Google Colab, Jupyter, LaTeX; Visual Studio; Eclipse; Sublime Text; Github; Unity Tool; Blender; Flask; Heroku; Bootstrap
			\end{itemize}
		\end{cvsubsection}
	\end{cvsection}
	
	\begin{cvsection}{Work Experience}
		\begin{cvsubsection}{Mozilla Builders Fix-The-Internet Open Lab - Developer}{}{April 2020 -- June 2020}
			\textit{Internship || Python, HTML/CSS/Javascript, Flask, Heroku}
			\begin{itemize}
			    \item Developed an online platform for matching donation related resources like food, clothing etc.using relevant tweets. The website lists donation/request tweets location wise, based on search. \underline{\href{https://help-for-all.herokuapp.com/}{Website Link}}
			    \item Implemented Naive Bayes Classifier for classification of tweets (Donation/Non-Donation, Donor/Requestor \& Resource Type classification), with an accuracy of 80\%, after parsing them using standard NLP techniques. (\href{https://github.com/shanujshekhar/MozillaOpenLab}{\underline{Project Link}})
			\end{itemize}
		\end{cvsubsection}
		
		\begin{cvsubsection}{Cadence Design Systems - Summer Intern}{}{June 2018}	    \textit{Text Detection in Images || C++, Microsoft Visual Studio 2017}
			\begin{itemize}
				\item Extracted text from Microprocessor Pin Diagram images. Using Posterior Probability concept text accuracy was improved
			\end{itemize}
		\end{cvsubsection}
		
		\begin{cvsubsection}{NSUT, Delhi - Research Intern}{}{June 2017}		
		    \textit{Reusable Hybrid Test Automation Framework for Web Based Scrum Project || Java, Selenium Tool 2.0}
			\begin{itemize}
				\item Achieved Automation Testing on Amazon, Flipkart e-commerce websites || \textbf{Published} in \textbf{Journal of Applied Science and Engineering}, Taiwan, 2018 (\href{http://www2.tku.edu.tw/~tkjse/21-3/catology.htm}{\underline{Publication Link}})
			\end{itemize}
		\end{cvsubsection}
	\end{cvsection}
	
	\begin{cvsection}{Technical Experience (Projects)}
		\begin{cvsubsection}{}{}{}
			\begin{itemize}
    			\item \textbf{Smart IoT Climate Control System} (Jan 2020 -- Ongoing). Currently developing a smart IoT climate control system by leveraging machine learning techniques for damper actuation (when to turn on heating/cooling) || Deep Neural Networks, PyTorch (\href{https://github.com/shanujshekhar/Smart-IOT-Climate-Control-System}{\underline{Project Link}})
    	    \end{itemize}
		\end{cvsubsection}
    	\begin{cvsubsection}{}{}{}
			\begin{itemize}
				\item \textbf{D3 Visualization of COVID-19 Pandemic} (Mar 2020). Created a dashboard for visualizing COVID-19 cases in the USA, how the disease spread and how it affected the country’s unemployment rates || Python, D3.js, Flask (\href{https://github.com/shanujshekhar/CoronaVIZ}{\underline{Project Link}})
			\end{itemize}
		\end{cvsubsection}
	    \begin{cvsubsection}{}{}{}
			\begin{itemize}
				\item \textbf{Detect Heavy Drinking Episodes} (Feb 2020). Implemented Random Forest Classifier to identify intoxicated individuals according to their TAC labels and detect drinking episodes using accelerometer samples from their mobile devices || Python (\href{https://github.com/shanujshekhar/Detect_Heavy_Drinking_Episodes}{\underline{Project Link}})
			\end{itemize}
		\end{cvsubsection}
		\begin{cvsubsection}{}{}{}
			\begin{itemize}
				\item \textbf{Augmented Reality Video Game} (Jan 2020). Designed a game in which a user can interactively build an augmented 3D scene on a planar surface in the real world || C\#, Unity Tool, Vuforia (\href{https://github.com/shanujshekhar/Augmented_Reality_Video_Game_using_Unity_Tool}{\underline{Project Link}})
			\end{itemize}
		\end{cvsubsection}
	    \begin{cvsubsection}{}{}{}
			\begin{itemize}
				\item \textbf{Generating entity descriptors (Post-Modifiers) based on context} (Nov 2019). Generated phrases to describe an entity in a sentence using Natural Language Processing and Natural Language Understanding. Model Architecture - Bi-LSTM (2 layer) model with attention function || Python (\href{https://github.com/shanujshekhar/Improvement-on-Knowledge-backed-Generation-Model-Using-Post-Modifier-Dataset}{\underline{Project Link}})
			\end{itemize}
		\end{cvsubsection}
	    \begin{cvsubsection}{}{}{}
			\begin{itemize}
				\item \textbf{Emotion Recognition} (Jun 2019). Performed facial expression analysis in near real-time live webcam feed \& classified 8 different emotions using Support Vector Machine with accuracy of 67\% || Python, OpenCV (\href{https://github.com/shanujshekhar/Emotion_Recognition}{\underline{Project Link}})
			\end{itemize}
		\end{cvsubsection}
	    \begin{cvsubsection}{}{}{}
			\begin{itemize}
				\item \textbf{TFIDF} (Feb 2017). Calculated the term frequency for terms present in 2000 documents || Java (\href{https://github.com/shanujshekhar/TFIDF}{\underline{Project Link}})
			\end{itemize}
		\end{cvsubsection}
	\end{cvsection}
	
	\begin{cvsection}{Relevant Coursework}
		\begin{cvsubsection}{}{}{}	
			\begin{itemize}
				\item ML, Visualization, NLP, OS, Virtual Reality, Data Mining, Data Structures \& Analysis of Algorithms, AI, Advanced Programming in Java
			\end{itemize}
		\end{cvsubsection}
	\end{cvsection}
	
	\begin{cvsection}{Additional Experience and Awards}
		\begin{cvsubsection}{}{}{}	
			\begin{itemize}
			    \item Completed \textbf{JP Morgan \& Chase Software Engineering Virtual Experience} (Summer 2020) (\href{https://github.com/shanujshekhar/JPMorgan_Chase_Co_Virtual_Experience}{\underline{Project Link}})
				\item Acquired \textbf{Top 10 Rank} in Undergrad in class of Computer Science
			\end{itemize}
		\end{cvsubsection}
	\end{cvsection}
	
\end{document}